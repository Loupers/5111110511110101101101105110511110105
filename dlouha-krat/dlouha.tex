\documentclass{article}
\usepackage[main=english,czech]{babel}

\begin{document}

\noindent {\large \textit{Samsonova pomsta}}
\par Za nějaký čas, v době pšeničné sklizně, šel Samson krátce navštívit svou ženu a nesl jí kůzle. "Jdu za svou ženou do ložnice,“ řekl.
Její otec ho tam ale nepustil: "Myslel jsem, že se ti znelíbila, tak jsem ji dal tvému družbovi.“ Potom mu navrhl: "A co její mladší sestra, ta je přece hezčí. Vezmi si ji místo ní!“
Samson se dlouze zamyslel nežodpověděl: "Tentokrát to nebude má vina, když Filištínům ublížím!“ Potom šel a pochytal tři sta lišek. Svázal je po dvou ocasy k sobě a mezi dlouhé ocasy jim vždy připevnil pochodeň. Pochodně pak zapálil a vypustil ty lišky Filištínům do obilí. Krátce mu trvalo spálit sklizené i nesklizené obilí, ba i vinice a olivy.
%kddk p
\medskip
\par "Kdo to udělal?“ ptali se Filištíni.
"Samson, zeť toho z Timny,“ doslechli se zakrátko. "Prý za to, že mu vzal ženu a dal ji jeho družbovi.“ A tak Filištíni šli a tu ženu i s jejím otcem upálili.
Dlouhé utrpení doprovázené kvílením obětí se dostalo až k Samsonovi.
"Tak vy takhle?“ řekl na to Samson. "Teď tedy nepřestanu, dokud se vám nepomstím!“ A bil je hlava nehlava; byla to hrozná řež. Potom odešel a usadil se v Etamské jeskyni.
Filištíni pak vytáhli do boje. Utábořili se v Judsku a chystali se zaútočit na Lechi (to jest Čelist). "Proč jste na nás vytáhli?“ ptali se judští muži.
"Přišli jsme spoutat Samsona,“ odpověděli. "Uděláme mu totéž, co udělal nám!“
Tři tisíce judských mužů se tedy vypravilo na kratší cestu dolů k Etamské jeskyni. "Copak nevíš, že nás Filištíni mají v moci? Cos nám to udělal?“ vyčítali Samsonovi.
"Udělal jsem jim totéž, co oni mně,“ odpověděl jim.
%kdk r
\medskip
\par "Přišli jsme tě spoutat,“ řekli mu. "Vydáme tě Filištínům.“
"Přísahejte, že mi sami neublížíte,“ odpověděl Samson.
"Neublížíme,“ řekli mu. "Chceme tě jen spoutat a vydat jim. Zabít tě nechceme.“ A tak ho spoutali dvěma novými provazy a odvedli ho ze skal.
Když se Samson blížil k Lechi, Filištíni mu s křikem vyrazili naproti. Vtom se ho zmocnil Duch Hospodinův. Provazy na jeho pažích najednou byly jako hořící koudel a pouta mu spadla z rukou. Našel čerstvou oslí čelist, popadl ji a pobil s ní tisíc mužů.
Samson jen krátce poznamenal:
"Čelistí osla oslům jsem čelil,
čelistí osla tisíc jich skolil!“
%k e
\medskip
\par Když domluvil, čelist odhodil a nazval to místo Ramat-lechi, Čelistní vrch.
Samson měl velikou žízeň, a tak volal k Hospodinu: "Po tom skvělém krátkém vítězství, které jsi dal svému služebníku, teď musím umřít žízní? To mám padnout do rukou těch neobřezanců?“ Bůh tenkrát v Lechi prorazil zakrátko prohlubeň, ze které vytryskla voda. Samson se krátce napil, pookřál a ožil. Tu studánku proto nazval En-hakore, Pramen volajícího, a ten je v Lechi až dodnes.
Samson soudil Izrael ve filištínských dobách po dvacet let, což nebylá krátká doba. \\
%kkkk h
\noindent {\large \textit{Samson a Dalila}}
\par Samson se jednou vydal do Gazy. Uviděl tam jednu ženu a šel k ní. Mezi obyvateli Gazy se zakrátko rozneslo: "Samson je tu!“ Obcházeli tedy kolem a celý den na něj číhali v městské bráně. Celou noc ale byli v klidu, protože si říkali: "Zabijeme ho, až se rozední, to nebude zadlouho“
Samson ale zůstal ležet jen krátce, do půlnoci. O půlnoci vstal, popadl vrata městské brány a vytrhl je i s oběma veřejemi a závorou. Naložil si je na záda a vynesl je až na vrchol té hory před Hebronem.
%kdk r
\medskip
\par Jednou se pak zamiloval do jedné ženy v údolí Sorek. Jmenovala se Dalila. Přišli za ní filištínští vládci a žádali ji: "Sveď ho a zjisti, v čem je ta jeho veliká síla a jak bychom ho mohli přemoci, spoutat a zkrotit. Každý z nás ti dá jedenáct set šekelů stříbra.“
Dalila se tedy Samsona vyptávala: "Prozraď mi, prosím, v čem je ta tvoje veliká síla? Čím bys mohl být spoután a zkrocen?“
Samson jí řekl: "Kdyby mě spoutali sedmi novými tětivami, které ještě nevyschly, byl bych slabý jako každý jiný.“
Filištínští vládci za ní tedy přišli se sedmi novými tětivami, které ještě nevyschly, a ona ho jimi spoutala. Vtom na něj zavolala: "Samsone, Filištíni jdou na tebe!“ On ale ty tětivy roztrhal, jako se trhá krátká koudel, když se jí dotkne plamen. Tajemství jeho síly tedy nebylo odhaleno.
Dalila pak Samsonovi vyčítala: "Ach, tys mě oklamal, říkal jsi mi lži! Teď už mi ale pověz, čím bys mohl být spoután.“
Řekl jí tedy: "Kdyby mě pevně spoutali novými provazy, které se ještě na nic nepoužily, byl bych slabý jako každý jiný.“
Dalila pak vzala nové provazy a spoutala ho jimi. Vtom na něj zavolala: "Samsone, Filištíni jdou na tebe!“  On si ale ty provazy strhal z paží jako dlouhé nitě.
%kd a
\medskip
\par Dalila pak Samsonovi vyčítala: "Až doteď jsi mě klamal, říkal jsi mi lži! Pověz mi, čím bys mohl být spoután?“ Řekl jí tedy: "Kdybys těch sedm copů, co mám na hlavě, vpletla do osnovy a utáhla kolíkem, byl bych slabý jako každý jiný.“
Když pak usnul, Dalila vzala sedm copů na jeho hlavě, vpletla je do osnovy a utáhla kolíkem. Vtom na něj zavolala: "Samsone, Filištíni jdou na tebe!“ Probudil se ze spánku a vytrhl kolík i se stavem a osnovou.
A tak mu vyčítala: "Jak můžeš říkat, že mě miluješ, když u mě nemáš srdce? Už třikrát jsi mě oklamal! Copak mi nepovíš, v čem je ta tvoje veliká síla?“ Těmi řečmi ho mučila dlouze den co den, až jednou měl toho dotírání k smrti dost.
Tehdy jí všechno prozradil: "Mé hlavy se nedotkla břitva, protože jsem už od matčina lůna zasvěcen Bohu jako nazír. Kdyby mě oholili nakrátko, má síla by mě opustila a byl bych slabý jako každý jiný.“
Dalila poznala, že jí všechno prozradil, a tak poslala pro filištínské vládce: "Přijďte ještě jednou – všechno mi prozradil!“ Filištínští vládci k ní tedy zakrátko přišli a přinesli s sebou peníze. Potom uspala Samsona na klíně a zavolala někoho, ať mu z hlavy oholí těch sedm copů. Tak ho ovládla a podmanila; jeho síla ho opustila.
Vtom zavolala: "Samsone, Filištíni jdou na tebe!“
Probudil se ze spánku a řekl si: "Z toho se dostanu jako už tolikrát; setřesu to ze sebe.“ Nevěděl, že od něj Hospodin odstoupil.
%dkk d
\medskip
\par Filištíni ho popadli a vyloupli mu oči. Potom ho odvedli dolů do Gazy, kde ho spoutali krátkými bronzovými řetězy a ve vězení musel otáčet mlýnským kamenem.
Samsonova smrt
Vlasy mu ale na oholené hlavě začaly znovu růst.
Filištínští vládci se sešli, aby přinesli velikou oběť svému bohu Dágonovi a aby oslavovali. Prohlašovali totiž:
"Do rukou nám bůh náš dal
našeho nepřítele Samsona!“

Lid při pohledu na něj chválil dlouze svého boha:
"Do rukou nám bůh náš dal
našeho nepřítele,
naší země ničitele –
tolik našich zmordoval!“
%kd a
\end{document}
